\section{Tabla de verdad con expresión booleana}

\subsection{Definición}
Los teoremas booleanos son reglas o propiedades que ayudan a simplificar expresiones y resolver problemas con lógica, los operadores booleanos son And, Or y Not. \cite{Boole}

\subsection{Descripción de el problema}
Dada una tabla de verdad de n bits generar la expresión booleana que genere de manera fidedigna las salidas de esta tabla.

\subsection{Diseño de la solución}
Con ayuda de los teoremas booleanos el programa deberá pedir al usuario ingresar un número de bits para crear las salidas de la tabla de verdad al igual que la expresión booleana basándose en la misma.


\subsection{Desarrollo de la solución}
Para la solución del problema se creó un programa en el cual se le pide al usuario que proporcione un número de bits y a su vez crear una tabla con el número de bits dado y así crear una expresión booleana con los resultados obtenidos.  


\begin{javaCode}
import java.util.Scanner;
import java.util.HashSet;
import java.util.Set;

public class tabla_de_verdad {
    public static void main(String[] args) {
        Scanner scanner = new Scanner(System.in);  // Se crea un objeto Scanner para la entrada del usuario.
        Set<Integer> filasCambiar = new HashSet<>();  // Se crea un conjunto para almacenar las filas que el usuario desea cambiar.

        System.out.print("Ingresa el numero de bits ");  // Se solicita al usuario el número de bits.
        int numBits = scanner.nextInt();  // Se lee el número de bits proporcionado por el usuario.

        // IMPRIMIR TABLA 
        for (int i = 0; i < numBits; i++) {
            System.out.print((char) ('A' + i) + "\t");  // Se imprimen los encabezados de las variables (A, B, C, etc.).
        }
        System.out.println("Resultado");  // Se imprime el encabezado de la columna "Resultado".

        for (int i = 0; i < Math.pow(2, numBits); i++) {
            imprimirFilaTabla(i, numBits);  // Se imprime una fila de la tabla de verdad original.
            System.out.println("0");  // Se imprime el resultado, que es siempre 0 en la tabla original.
        }
        // IMPRIMIR TABLA 

        while (true) {  // Bucle para Cambiar Bits
            System.out.print("Ingrese el número de la fila que desea cambiar el resultado a 1 (1-" + (int) Math.pow(2, numBits) + ") si desea finalizar ingresar 0: ");
            int fila = scanner.nextInt();  // Se lee la fila que el usuario desea cambiar.

            if (fila == 0) { break; }  // Si el usuario ingresa 0, se sale del bucle.

            if (fila < 1 || fila > Math.pow(2, numBits)) {  // Se valida que la fila esté en el rango válido.
                System.out.println("Número de fila inválido. Debe estar entre 1 y " + (int) Math.pow(2, numBits) + ".");
                continue;  // Se vuelve al inicio del bucle si la fila no es válida.
            }
            
            filasCambiar.add(fila);  // Se agrega la fila al conjunto de filas a cambiar.
            imprimirTablaActualizada(filasCambiar, numBits);  // Se imprime la tabla actualizada.
        }

        // Resultado BOOLEANO 
        System.out.println("Expresiones booleanas al finalizar:");
        filasCambiar.forEach(fila -> System.out.println(generarExpresionBooleana(fila, numBits)));  // Se imprimen las expresiones booleanas de las filas modificadas.
        System.out.println("Expresión booleana final: " + generarExpresionFinal(filasCambiar, numBits));  // Se imprime la expresión booleana final.
        // Resultado BOOLEANO 
    }

    private static void imprimirFilaTabla(int valor, int numBits) {
        for (int j = numBits - 1; j >= 0; j--) {
            System.out.print(((valor >> j) & 1) + "\t");  // Se imprime cada bit de la fila en formato binario.
        }
    }

    private static void imprimirTablaActualizada(Set<Integer> filasCambiar, int numBits) {
        System.out.println("Tabla de verdad actualizada:");
        for (int i = 0; i < Math.pow(2, numBits); i++) {
            imprimirFilaTabla(i, numBits);
            boolean resultado = false;
            if (filasCambiar.contains(i + 1)) {
                resultado = true;
            }
            System.out.println(resultado ? "1" : "0");  // Se imprime el resultado (1 o 0) de acuerdo a las filas cambiadas.
        }
    }

     // Función para generar expresiones booleanas
    private static String generarExpresionBooleana(int fila, int numBits) {
        StringBuilder expresion = new StringBuilder();
        for (int i = 0; i < numBits; i++) {
            expresion.append(((fila - 1) >> (numBits - 1 - i) & 1) == 1 ? (char) ('A' + i) : (char) ('A' + i) + "'");
            if (i < numBits - 1) {
                expresion.append(" + ");
            }
        }
        return expresion.toString();  // Se genera la expresión booleana para una fila específica.
    }

    private static String generarExpresionFinal(Set<Integer> filasCambiar, int numBits) {
        StringBuilder expresionesConcatenadas = new StringBuilder();
        filasCambiar.forEach(fila -> {
            String expresion = generarExpresionBooleana(fila, numBits);
            if (expresionesConcatenadas.length() > 0) {
                expresionesConcatenadas.append(" + ");
            }
            expresionesConcatenadas.append
            ("(").append(expresion).append(")");
        });
        return expresionesConcatenadas.toString();  
        // Se genera la expresión booleana final concatenando las expresiones de las filas modificadas.
    }
}
     }
}
\end{javaCode}

\vspace{4cm}
\subsection{Depuracion y pruebas}
En el cuadro 3 se muestran los resultados arrojados al compilar el código.

\begin{table}[!ht]
\label{T:equipos}
\begin{center}
\begin{tabular}{| c | c | c | c | c | c |}
\hline
\textbf{$No$} & \textbf{$A$} &  \textbf{$B$} & \textbf{$C$} & \textbf{$RESULTADO$}\\
\hline
1 & 0 & 0 & 0 & 1\\
2 & 0 & 0 & 1 & 1 \\
3 & 0 & 1 & 0 &  0\\
4 & 0 & 1 & 1 & 1 \\
5 & 1 & 0 & 0 & 0 \\
6 & 1 & 0 & 1 & 1 \\
7 & 1 & 1 & 0 & 1  \\
8 & 1 & 1 & 1 & 1 \\
\hline
\end{tabular}
\caption{Tabla de corridas.}
\end{center}
\end{table}\\




