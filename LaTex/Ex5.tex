
\section{Binario a decimal}

\subsection{Definición}

El problema consiste en convertir un número binario de $n$ bits a su equivalente en decimal. Se desea desarrollar un programa que realice esta conversión de manera eficiente y precisa.

\subsection{Descripción del problema}

Un número binario está compuesto únicamente por los dígitos 0 y 1, y su representación en decimal se basa en el sistema numérico posicional, donde cada dígito tiene un valor asociado dependiendo de su posición. El bit más a la derecha tiene un valor de $2^0$, el siguiente bit tiene un valor de $2^1$, el siguiente $2^2$, y así sucesivamente. Para convertir un número binario a decimal, se deben multiplicar los dígitos binarios por las potencias de 2 correspondientes y luego sumar los resultados.

Por ejemplo, el número binario 1011 se convierte a decimal de la siguiente manera:
\begin{align*}
1 \cdot 2^3 + 0 \cdot 2^2 + 1 \cdot 2^1 + 1 \cdot 2^0 = 11
\end{align*}

El objetivo es desarrollar un programa que tome como entrada un número binario de $n$ bits y devuelva su equivalente en decimal.

\subsection{Diseño de solución}

El diseño de la solución implica el desarrollo de un algoritmo que realice la conversión del número binario a decimal. El algoritmo seguirá los siguientes pasos:

\begin{enumerate}
  \item Solicitar al usuario que ingrese un número binario de $n$ bits.
  \item Inicializar una variable \texttt{decimal} en 0 para almacenar el resultado de la conversión.
  \item Recorrer el número binario de derecha a izquierda, comenzando desde el bit más a la derecha:
    \begin{itemize}
      \item Obtener el dígito binario en la posición actual.
      \item Multiplicar el dígito por $2^i$, donde $i$ es la posición actual.
      \item Sumar el resultado a la variable \texttt{decimal}.
    \end{itemize}
  \item Mostrar el resultado \texttt{decimal}, que es el número decimal equivalente al número binario ingresado.
\end{enumerate}

\subsection{Desarrollo de solución}

A continuación se presenta la implementación del algoritmo en el lenguaje de programación JAVA:

\begin{javaCode}
import java.util.Scanner;
public class bin {

    public static void main(String[] args) {
        // entrada
        Scanner entrada = new Scanner(System.in);
        System.out.println("Ingresa un numero binario");
        String numeroBinario=entrada.nextLine();
        
        int longitud=numeroBinario.length();
        
        int numeroDecimal=0;
        
        for(int i=0; i<longitud; i++){
            char digito=numeroBinario.charAt(i);
            //verificar si es 0 o 1
            if(digito=='0'){
                numeroDecimal= numeroDecimal*2;
                
            
            }else if(digito =='1'){
                numeroDecimal= numeroDecimal*2+1;
            }else{
                System.out.println("El numero binario ingresado no es valido");
                return;
            }
        }
        System.out.println("El numero decimal equivalente es:"+ numeroDecimal);
    }
}

\end{javaCode}

El programa solicita al usuario que ingrese un número binario y luego utiliza la función \texttt{binario\_a\_decimal()} para convertirlo a su equivalente en decimal. El resultado se muestra en pantalla.

\subsection{Depuración y prueba}

Durante el proceso de desarrollo y posterior ejecución del programa, pueden surgir errores o problemas. Para depurar el programa, se pueden utilizar técnicas como la impresión de mensajes de depuración, el uso de herramientas de depuración y la revisión completa del código en busca de posibles errores.\\

\begin{tabular}{|c|c|c|}
  \hline
  \textbf{Entrada} & \textbf{Salida Esperada} & \textbf{Salida Obtenida} \\
  \hline
  1011 & 11 & 11 \\
  \hline
  1101 & 13 & 13 \\
  \hline
  10001 & 17 & 17 \\
  \hline
  1111 & 15 & 15 \\
  \hline
\end{tabular}
\caption{Compilaciones.}
