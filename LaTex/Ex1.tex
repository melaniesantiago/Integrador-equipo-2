
\section{Ecuación de la recta}
\subsection{Definición}
Dados 2 puntos A y B con coordenadas $(x1, y1)$ y
$(x2, y2)$ respectivamente. Regresar la ecuación de la recta y el ángulo interno $\alpha$ que se forma entre el eje horizontal y la recta.

\subsection{Descripción del problema}

En este problema, se tienen dos puntos $A$ y $B$ con coordenadas $(x_1, y_1)$ y $(x_2, y_2)$ respectivamente. El objetivo es calcular la ecuación de la recta que pasa por estos puntos y determinar el ángulo interno $\alpha$ que dicha recta forma con el eje horizontal.

\subsection{Diseño de la solución}

La ecuación de la recta se obtiene utilizando la fórmula de la pendiente:

\begin{equation}
m = \frac{y_2 - y_1}{x_2 - x_1}
\end{equation}

Una vez obtenida la pendiente $m$, el ángulo interno $\alpha$ se puede calcular utilizando la función trigonométrica arcotangente:

\begin{equation}
\alpha = \arctan(m)
\end{equation}

\subsection{Desarrollo de la solución}

Para resolver el problema, se puede utilizar un programa en JAVA. A continuación se muestra un ejemplo de código:

\begin{javaCode}
import java.util.Scanner;

public class recta {
    public static void main(String[] args) {
        Scanner scanner = new Scanner(System.in);

        System.out.println("Ingrese las coordenadas del punto A:");
        System.out.print("x1: ");
        double x1 = scanner.nextDouble();
        System.out.print("y1: ");
        double y1 = scanner.nextDouble();

        System.out.println("Ingrese las coordenadas del punto B:");
        System.out.print("x2: ");
        double x2 = scanner.nextDouble();
        System.out.print("y2: ");
        double y2 = scanner.nextDouble();

        // Calcular la pendiente de la recta
        double pendiente = (y2 - y1) / (x2 - x1);

        // Calcular el ángulo en radianes
        double anguloRadianes = Math.atan(pendiente);

        // Convertir el ángulo a grados
        double anguloGrados = Math.toDegrees(anguloRadianes);

        // Calcular el ángulo interno α entre el eje horizontal y la recta
        double anguloInterno = 90 - anguloGrados;

        // Calcular el punto de intersección con el eje y (cuando x = 0)
        double puntoInterseccionY = y1 - pendiente * x1;

        // Construir la ecuación de la recta en formato y = mx + b
        String ecuacionRecta = "y = " + pendiente + "x + " + puntoInterseccionY;

        System.out.println("Ecuación de la recta: " + ecuacionRecta);
        System.out.println("Ángulo interno α: " + anguloInterno + "grados");
        System.out.println("Punto de intersección con el eje y: (0, " + puntoInterseccionY + ")");
        System.out.println("Pendiente = "+pendiente);
       
    }
}
\end{javaCode}

\subsection{Depuración y pruebas}

Durante el desarrollo y ejecución del programa, es posible encontrar errores o problemas. Para depurar el programa, se pueden utilizar técnicas como la impresión de mensajes de depuración, el uso de herramientas de depuración y la revisión del código en busca de posibles errores. Luego de depurar el programa, se puede proceder a realizar pruebas para verificar su funcionamiento correcto.

En el cuadro 1 se muestran los resultados obtenidos al compilar el código.\\

\begin{table}[!ht]
\label{T:equipos}
\begin{center}
\begin{tabular}{| c | c | c | c | c | c | c |}
\hline
\textbf{$x_1$} & \textbf{$x_2$} & \textbf{$y_1$} & \textbf{$y_2$} & \textbf{Pendiente} & \textbf{Inclinación} & \textbf{Intersección} \\
\hline
2 & 4 & 4 & 6 & 1 & 45$^o$ & -2 \\
3 & 2 & 4 & 1 & 3 & 71.57$^o$ & -5 \\
2 & 2 & 2 & 2 & 0 & 0$^o$ & 0 \\
1 & 3 & 2 & 4 & 1 & 45$^o$ & 1 \\
\hline
\end{tabular}
\caption{Tabla de corridas.}
\end{center}
\end{table}